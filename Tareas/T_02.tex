  \documentclass[10pt]{article}\usepackage[]{graphicx}\usepackage[]{color}
% maxwidth is the original width if it is less than linewidth
% otherwise use linewidth (to make sure the graphics do not exceed the margin)
\makeatletter
\def\maxwidth{ %
  \ifdim\Gin@nat@width>\linewidth
    \linewidth
  \else
    \Gin@nat@width
  \fi
}
\makeatother

\definecolor{fgcolor}{rgb}{0.345, 0.345, 0.345}
\newcommand{\hlnum}[1]{\textcolor[rgb]{0.686,0.059,0.569}{#1}}%
\newcommand{\hlstr}[1]{\textcolor[rgb]{0.192,0.494,0.8}{#1}}%
\newcommand{\hlcom}[1]{\textcolor[rgb]{0.678,0.584,0.686}{\textit{#1}}}%
\newcommand{\hlopt}[1]{\textcolor[rgb]{0,0,0}{#1}}%
\newcommand{\hlstd}[1]{\textcolor[rgb]{0.345,0.345,0.345}{#1}}%
\newcommand{\hlkwa}[1]{\textcolor[rgb]{0.161,0.373,0.58}{\textbf{#1}}}%
\newcommand{\hlkwb}[1]{\textcolor[rgb]{0.69,0.353,0.396}{#1}}%
\newcommand{\hlkwc}[1]{\textcolor[rgb]{0.333,0.667,0.333}{#1}}%
\newcommand{\hlkwd}[1]{\textcolor[rgb]{0.737,0.353,0.396}{\textbf{#1}}}%
\let\hlipl\hlkwb

\usepackage{framed}
\makeatletter
\newenvironment{kframe}{%
 \def\at@end@of@kframe{}%
 \ifinner\ifhmode%
  \def\at@end@of@kframe{\end{minipage}}%
  \begin{minipage}{\columnwidth}%
 \fi\fi%
 \def\FrameCommand##1{\hskip\@totalleftmargin \hskip-\fboxsep
 \colorbox{shadecolor}{##1}\hskip-\fboxsep
     % There is no \\@totalrightmargin, so:
     \hskip-\linewidth \hskip-\@totalleftmargin \hskip\columnwidth}%
 \MakeFramed {\advance\hsize-\width
   \@totalleftmargin\z@ \linewidth\hsize
   \@setminipage}}%
 {\par\unskip\endMakeFramed%
 \at@end@of@kframe}
\makeatother

\definecolor{shadecolor}{rgb}{.97, .97, .97}
\definecolor{messagecolor}{rgb}{0, 0, 0}
\definecolor{warningcolor}{rgb}{1, 0, 1}
\definecolor{errorcolor}{rgb}{1, 0, 0}
\newenvironment{knitrout}{}{} % an empty environment to be redefined in TeX

\usepackage{alltt}  

%%%%%%%% PREÁMBULO %%%%%%%%%%%%
\title{Plantilla para reportes IMEC-UTB}
\usepackage[spanish]{babel} %Indica que escribiermos en español
\usepackage[utf8]{inputenc} %Indica qué codificación se está usando ISO-8859-1(latin1)  o utf8  
\usepackage{amsmath} % Comandos extras para matemáticas (cajas para ecuaciones,
% etc)
\usepackage{rotating}%rotar mis tablasw
\usepackage{amssymb} % Simbolos matematicos (por lo tanto)
\usepackage{graphicx} % Incluir imágenes en LaTeX
\usepackage{color} % Para colorear texto
\usepackage{subfigure} % subfiguras
\usepackage{float} %Podemos usar el especificador [H] en las figuras para que se
% queden donde queramos
\usepackage[T1]{fontenc}
\usepackage{pdfsync}
\usepackage[nottoc]{tocbibind}
\usepackage{amsfonts}
\usepackage{multicol}
\usepackage{changepage}
\usepackage{float}
\usepackage{cite}
\usepackage{url}
\usepackage[export]{adjustbox}
\usepackage{pstricks}
\usepackage{capt-of} % Permite usar etiquetas fuera de elementos flotantes
% (etiquetas de figuras)
\usepackage{sidecap} % Para poner el texto de las imágenes al lado
	\sidecaptionvpos{figure}{c} % Para que el texto se alinie al centro vertical
\usepackage{caption} % Para poder quitar numeracion de figuras
\usepackage{commath} % funcionalidades extras para diferenciales, integrales,
% etc (\od, \dif, etc)
\usepackage{cancel} % para cancelar expresiones (\cancelto{0}{x})
\usepackage{anysize}% Para personalizar el ancho de  los márgenes
\marginsize{2cm}{2cm}{2cm}{2cm} % Izquierda, derecha, arriba, abajo
\usepackage{appendix}
\renewcommand{\appendixname}{Apéndices}
\renewcommand{\appendixtocname}{Apéndices}
\renewcommand{\appendixpagename}{Apéndices} 

% Para que las referencias sean hipervínculos a las figuras o ecuaciones y
% aparezcan en color
\usepackage[colorlinks=true,plainpages=true,citecolor=blue,linkcolor=blue]{hyperref}
% Para agregar encabezado y pie de página
\usepackage{fancyhdr} 
\pagestyle{fancy}
\fancyhf{}
\fancyhead[L]{\footnotesize IPN} %encabezado izquierda
\fancyhead[R]{\footnotesize ESFM}   % dereecha
\fancyfoot[R]{\footnotesize Tarea 02}  % Pie derecha
\fancyfoot[C]{\thepage}  % centro
\fancyfoot[L]{\footnotesize Ingeniería Matemática }  %izquierda
\renewcommand{\footrulewidth}{0.4pt}


\usepackage{listings} % Para usar código fuente
\definecolor{dkgreen}{rgb}{0,0.6,0} % Definimos colores para usar en el código
\definecolor{gray}{rgb}{0.5,0.5,0.5} 
% configuración para el lenguaje que queramos utilizar
\lstset{language=Matlab,
   keywords={break,case,catch,continue,else,elseif,end,for,function,
      global,if,otherwise,persistent,return,switch,try,while},
   basicstyle=\ttfamily,
   keywordstyle=\color{blue},
   commentstyle=\color{red},
   stringstyle=\color{dkgreen},
   numbers=left,
   numberstyle=\tiny\color{gray},
   stepnumber=1,
   numbersep=10pt,
   backgroundcolor=\color{white},
   tabsize=4,
   showspaces=false,
   showstringspaces=false}
\newcommand{\sen}{\operatorname{\sen}}	

\title{Plantilla para tareas ESFM}

%%%%%%%% TERMINA PREÁMBULO %%%%%%%%%%%%

\IfFileExists{upquote.sty}{\usepackage{upquote}}{}
\begin{document}

%%%%%%%%%%%%%%%%%%%%%%%%%%%%%%%%%% PORTADA %%%%%%%%%%%%%%%%%%%%%%%%%%%%%%%%%%%%%%%%%%%%
																					%%%
\begin{center}																		%%%
\newcommand{\HRule}{\rule{\linewidth}{0.5mm}}									%%%\left
 																					%%%
\begin{minipage}{0.48\textwidth} \begin{flushleft}
\includegraphics[scale = 0.71]{IPN.JPG}
\end{flushleft}\end{minipage}
\begin{minipage}{0.48\textwidth} \begin{flushright}
\includegraphics[scale = 0.67]{ESFM.pdf}
\end{flushright}\end{minipage}

													 								%%%
\vspace*{1.0cm}								%%%
																					%%%	
\textsc{\huge Instituto Politécnico Nacional \\ \vspace{5px} Escuela Superior de Física y Matemáticas}\\[1.5cm]	

\textsc{\LARGE  Ingeniería Matemática \\ \vspace{5px} Línea Financiera }\\[1.5cm]													%%%


%%%
    																				%%%
 			\vspace*{1cm}																		%%%
																					%%%
\HRule \\[0.4cm]																	%%%
{ \huge \bfseries Tarea 02: Distribuciones de probabilidad}\\[0.4cm]	%%%
 																					%%%
\HRule \\[1 cm]																	%%%
 																				%%%
																					%%%
\begin{minipage}{0.46\textwidth}													%%%
\begin{flushleft} \large															%%%

% Aqui a continuación pongan los nombres de los integrantes
\emph{Alumno:}\\	
Yañez Perez Gabriel Osvaldo\\
Boleta: 2019330158\\
Correo electronico: gyanezp1500@alumno.ipn.mx\\
Grupo: 8MM1
%%%
			%\vspace*{2cm}	
            													%%%
										 						%%%
\end{flushleft}																		%%%
\end{minipage}		
																%%%
\begin{minipage}{0.52\textwidth}		
\vspace{-0.6cm}											%%%
\begin{flushright} \large															%%%
\emph{Profesor:} \\																	%%%
Medel Esquivel Ricardo\\
													%%%
\end{flushright}																	%%%
\end{minipage}	
\vspace*{1cm}
%\begin{flushleft}
 	
%\end{flushleft}
%%%
 		\flushleft{\textbf{\Large Simulación II}	}\\																		%%
\vspace{2cm} 																				
\begin{center}	
Ciudad de México  \\
{\large \today}																	%%%
 			\end{center}												  						
\end{center}							 											
																					
\newpage																			
%%%%%%%%%%%%%%%%%%%% TERMINA PORTADA %%%%%%%%%%%%%%%%%%%%%%%%%%%%%%%%

\tableofcontents 
\newpage

\section{Variables aleatorias discretas}

\subsection{Definición básica}
Se dice que una variable aleatoria \textit{Y} es discreta si puede tomar sólo un número finito o contablemente infinito de valores distintos.\\
De modo notacional usaremos una letra mayúscula, por ejemplo \textit{Y}, para denotar una \textbf{variable aleatoria} y una letra minúscula, por ejemplo \textit{y}, para denotar un \textbf{valor particular} que puede tomar una variable aleatoria.\\
La probabilidad de que \textit{Y} tome el valor \textit{y}, \textit{P(Y=y)}, se define como la suma de las probabilidades de todos los puntos muestrales en \textit{S} a los que se asigna el valor \textit{y}.\\
Una \textbf{distribución de probabilidad} para una variable discreta \textit{Y} puede ser representada por una fórmula, una tabla o una gráfica que produzca $p(y)=P(Y=y)$ para toda \textit{y}.\\
\textbf{Distribución de probabilidad para Y}\\
\begin{tabular}{p{0.900\textwidth} p{0.910\textwidth} }
\includegraphics[width=1\textwidth, center]{captura01.PNG}
\end{tabular}\\
Sea \textit{Y} una variable aleatoria discreta con la función de probabilidad \textit{p(y)}. Entonces el \textbf{valor esperado} de \textit{Y}, \textit{E(Y)}, se define como:
\begin{equation*}
E(Y)=\sum_y yp(y)
\end{equation*}
Si \textit{p(y)} es una caracterización precisa de la distribución de frecuencia poblacional, entonces $E(Y)=\mu$ es la \textbf{media poblacional}.\\
Si \textit{Y} es una variable aleatoria con media $E(Y)=\mu$, la varianza de una variable aleatoria \textit{Y} se define como el valor esperado de $(Y-\mu)^2$.
\begin{equation*}
V(Y)=E[(Y-\mu)^2]
\end{equation*}
La \textit{desviación estándar} de \textit{Y} es la raíz cuadrda positiva de \textit{V(Y)}
\begin{equation*}
V(Y)=\sigma^2
\end{equation*}
$\sigma^2$ es la varianza poblacional y $\sigma$ es la desviación estándar poblacional.


\subsection{La distribución de probabilidad binomial}
Se dice que una variable aleatoria \textit{Y} tiene una \textit{distribución binomial} basada en \textit{n} pruebas con probabilidad \textit{p} de éxito y si y sólo si.
\begin{equation*}
p(y)=\begin{pmatrix}
&n&\\
&y&
\end{pmatrix}
p^yq^{n-y}\; \; \; \; \; \; \; \; \; \;     y=0,1,2,...,n \;\; \;y \; \; \;0\leq p \geq 1
\end{equation*}\\
Sea \textit{Y} una variable aleatoria binomial basada en \textit{n} pruebas y probabilidad  \textit{p} de éxito.\\
Entonces
\begin{align*}
\mu&=E(Y)=np\\
\sigma^2&=V(Y)=npq
\end{align*}
Histogramas de probabilidad binomial.\\
\begin{tabular}{p{0.900\textwidth} p{0.910\textwidth} }
\includegraphics[width=1\textwidth, center]{captura02.PNG}
\end{tabular}\\
\begin{tabular}{p{0.900\textwidth} p{0.910\textwidth} }
\includegraphics[width=1\textwidth, center]{captura03.PNG}
\end{tabular}\\
La distribución de probabilidad binomial tiene muchas aplicaciones porque el experimento 
binomial se presenta al muestrear defectos en control de calidad industrial, en el muestreo de 
preferencia de los consumidores o en poblaciones de votantes y en muchas otras situaciones 
físicas.



\subsection{La distribución de probabilidad geométrica}
Se dice que una variable aleatoria \textit{Y} tiene una \textit{distribución de pobabilidad geométrica} si y sólo si.
\begin{equation*}
p(y)=q^{y-1}p\; \; \; \; \; \; \; \; \; \;     y=0,1,2,...,n \;\; \;y \; \; \;0\leq p \geq 1
\end{equation*}\\
Sea \textit{Y} una variable aleatoria con una distribución geométrica.\\
Entonces
\begin{align*}
\mu&=E(Y)=\frac{1}{p}\\
\sigma^2&=V(Y)=\frac{1-p}{p^2}
\end{align*}
\begin{tabular}{p{0.900\textwidth} p{0.910\textwidth} }
\includegraphics[width=1\textwidth, center]{captura04.PNG}
\end{tabular}\\
La distribución de probabilidad geométrica se emplea con frecuencia para modelar distribuciones de la duración de tiempos de espera. Por ejemplo, suponga que el motor de un 
avión comercial recibe atención periódicamente para cambiar sus diversas partes en puntos 
diferentes de tiempo y que, por tanto, son de edades que varían. Entonces la probabilidad p
de mal funcionamiento del motor durante cualquier intervalo de operación de una hora observado al azar podría ser el mismo que para cualquier otro intervalo de una hora. El tiempo 
transcurrido antes de que el motor falle es el número de intervalos de una hora, Y, hasta que 
ocurra la primera falla. (Para esta aplicación, el mal funcionamiento del motor en un periodo 
determinado de una hora se defi ne como éxito)



\subsection{La distribución de probabilidad hipergeométrica}
Se dice que una variable aleatoria $Y$ tiene \textit{distribución de probabilidad hipergeométrica} si y sólo si.
\begin{equation*}
p(y)=\frac{\begin{pmatrix}
&r&\\
&y&
\end{pmatrix}
\begin{pmatrix}
&N-r&\\
&n-y&
\end{pmatrix}}{
\begin{pmatrix}
&N&\\
&n&
\end{pmatrix}
}
\end{equation*}
donde $y$ es un entero 0,1,2,...,n,sujeto a las restricciones $y\leq r$ y $n-y\leq N-r$. 
Si $Y$ es una variable aleatoria con distribución hipergeométrica.\\
Entonces
\begin{align*}
\mu&=E(Y)=\frac{nr}{N}\\
\sigma^2&=V(Y)=n(\frac{r}{N})(\frac{N-r}{N})(\frac{N-n}{N-1})
\end{align*}



\subsection{La distribución de probabilidad Poisson}
Se dice que una variable aleatoria $Y$ tiene \textit{distribución de probabilidad de Poisson} si y sólo si.
\begin{equation*}
p(y)=\frac{\lambda^y}{y!}\exp{-\lambda}\; \; \; \; \; \; \; \; \;y=0,1,2,...\; \; \; \; \; \; \lambda>0
\end{equation*}
Si $Y$ es una variable aleatoria que posee una distribución de Poisson con parámetro $\lambda$, entonces
\begin{align*}
\mu&=E(Y)=\lambda\\
\sigma^2&=V(Y)=\lambda
\end{align*}
\begin{tabular}{p{0.900\textwidth} p{0.910\textwidth} }
\includegraphics[width=1\textwidth, center]{captura09.PNG}
\end{tabular}\\




\newpage
\subsection{Tabla de resumen}

\renewcommand{\arraystretch}{1.8}
\begin{tabular}[t]{|l |c |c |c |}
\hline
Distribucion de Probabilidad & Función de probabilidad p(Y) & Esperanza  & Varianza \\
\hline
Binomial  &
$p(y)=\begin{pmatrix}
&n&\\
&y&
\end{pmatrix}
p^yq^{n-y}$ & $\mu=E(Y)=np$ & 
$\sigma^2=V(Y)=npq$ \\
\hline

Geométrica & $p(y)=q^{y-1}p$ & $\mu=E(Y)=\frac{1}{p}$ &
$\sigma^2=V(Y)=\frac{1-p}{p^2}$  \\
\hline

Hipergeométrica & 
$p(y)=\frac{\begin{pmatrix}
&r&\\
&y&
\end{pmatrix}
\begin{pmatrix}
&N-r&\\
&n-y&
\end{pmatrix}}{
\begin{pmatrix}
&N&\\
&n&
\end{pmatrix}}$ & $\mu=E(Y)=\frac{nr}{N}$ &
$\sigma^2=V(Y)=n(\frac{r}{N})(\frac{N-r}{N})(\frac{N-n}{N-1})$\\
\hline


Poisson & $p(y)=\frac{\lambda^y}{y!}\exp{-\lambda}$  & $\mu=E(Y)=\lambda$ & $\sigma^2=V(Y)=\lambda $\\

\hline
\end{tabular}








\newpage
\section{Variables aleatorias continuas}
\subsection{Distribución de probabilidad para una variable aleatoria continua}
Denote con \textit{Y} cualquier variable aleatoria. \textbf{La función de distribución} de \textit{Y}, denotada por $F(y)$, es tal que $F(y)=P(Y\leq y)$ para $-\infty < y >\infty$\\
Sea $F(y)$ la función de destribucion para una variable aleatoria continua \textit{Y}. Entonces $f(y)$, dada por\\
\begin{equation*}
f(y)=\frac{dF(y)}{dy}=F'(y)
\end{equation*}\\
siempre que exista la derivada, se denomina \textbf{función de densidad deprobabilidad} para la variable aleatoria \textit{Y}.
\begin{equation*}
F(Y)=\int_{-\infty}^{y}f(t)dt
\end{equation*}
donde \textit{f()} es la función de desnsidad de la probabilidad y \textit{t} se usa como la variable de integración.\\
El valor esperado de una varible aleatoria continua \textit{Y} es
\begin{equation*}
E(Y)=\int_{-\infty}^{\infty}yf(y)dy
\end{equation*}



\subsection{La distribución de probabilidad uniforme}
Si $\theta_1 < \theta_2$ se dice que un variable aleatoria \textit{Y} tiene \textit{distribución de probabilidad uniforme} en el intervalo $(\theta_1,\theta_2)$ si y solo si la funión de densidad \textit{Y} es
\begin{equation*}
f(y)=\left\lbrace\begin{array}{c} \frac{1}{\theta_2 - \theta_1},~\theta_1\leq~ y ~\geq \theta_2
\\ 0, ~en~ cualquier~ otro ~punto \end{array}\right.
\end{equation*}\\
Si $\theta_1<\theta_2$ y $Y$ es una variable aleatoria uniformemente distribuida en el intervaldo $(\theta_1,\theta_2)$, entonces
\begin{align*}
\mu&=E(Y)=\frac{\theta_1 +\theta_2}{2}\\
\sigma^2&=V(Y)=\frac{(\theta_2-\theta_1)^2}{12}
\end{align*}
\begin{tabular}{p{0.900\textwidth} p{0.910\textwidth} }
\includegraphics[width=1\textwidth, center]{captura05.PNG}
\end{tabular}\\
Como veremos, la distribución uniforme es muy importante por razones teóricas. Los estudios de \textbf{simulación} son técnicas valiosas para validar modelos en estadística. Si deseamos 
un conjunto de observaciones de una variable aleatoria Y con función de distribución $F(y)$, a 
menudo podemos obtener los resultados deseados si transformamos un conjunto de observaciones en una variable aleatoria uniforme. Por esta razón, casi todos los sistemas de cómputo 
contienen un generador de números aleatorios que produce valores observados para una variable aleatoria que tiene una distribución uniforme continua





\subsection{La distribución de probabilidad normal}
Se dice que un variable aleatoria \textit{Y} tiene \textit{distribución normal de probabilidad } si y sólo si, para $\sigma >0$ y $-\infty<\mu<\infty$, la función de densidad de $Y$ es
\begin{equation*}
f(y)=\frac{1}{\sigma\sqrt{2\pi}}exp{-(y-\mu)^2/(2\sigma^2)}
\end{equation*}
Si $Y$ es una variable aleatoria normalmente distribuida con párametros $\mu$ y $\sigma$, entonces
\begin{align*}
E(Y)&=\mu \\
V(Y)&=\sigma^2
\end{align*}\\ 
\begin{tabular}{p{0.900\textwidth} p{0.910\textwidth} }
\includegraphics[width=1\textwidth, center]{captura06.PNG}
\end{tabular}\\
Esta distribución es frecuentemente utilizada en las
aplicaciones estadísticas. Su propio nombre indica su extendida utilización,
justificada por la frecuencia o normalidad con la que ciertos fenómenos tienden
a parecerse en su comportamiento a esta distribución.\\
Muchas variables aleatorias continuas presentan una
función de densidad cuya gráfica tiene forma de campana.\\
En otras ocasiones, al considerar distribuciones
binomiales, tipo B(n,p), para un mismo valor de p y valores
de n cada vez mayores, se ve que sus polígonos de
frecuencias se aproximan a una curva en "forma de campana".\\
En resumen, la importancia de la distribución normal se debe principalmente a que hay muchas variables asociadas a fenómenos naturales que siguen el modelo de la normal.
\begin{itemize}
\item[]$\blacklozenge$ Caracteres morfológicos de individuos (personas, animales, plantas,...) de una especie, p.ejm. tallas, pesos, envergaduras, diámetros, perímetros,...\\
\item[]$\blacklozenge$ Caracteres fisiológicos, por ejemplo: efecto de una misma dosis de un fármaco, o de una misma cantidad de abono.\\
\item[] $\blacklozenge$ Caracteres sociológicos, por ejemplo: consumo de cierto producto por un mismo grupo de individuos, puntuaciones de examen.
\item[]$\blacklozenge$ Caracteres psicológicos, por ejemplo: cociente intelectual, grado de adaptación a un medio,...
\item[]$\blacklozenge$ Otras distribuciones como la binomial o la de Poisson son aproximaciones normales, ...
\end{itemize}



\subsection{La distribución de probabilidad gamma}
Se dice que un variable aleatoria \textit{Y} tiene una \textit{distribución gamma} con parámetros $\sigma>0$ y $\beta>0$ si y sólo si la función de densidad de $Y$ es
\begin{equation*}
f(y)=\left\lbrace\begin{array}{c} \frac{y^{\sigma-1}\exp{-y/\beta}}{\beta^\sigma \Gamma(\sigma)},~0\leq~ y ~<\infty
\\ 0, ~en~ cualquier~ otro ~punto \end{array}\right.
\end{equation*}\\
donde
\begin{equation*}
\Gamma(\sigma)=\int_{0}^{\infty} y^{\sigma-1}\exp{(-y)}dy.
\end{equation*}
La cantidad $\Gamma(\sigma)$ se conoce como \textit{función gamma.}
\\Si $Y$ tiene una distribución gamma con parámetros $\sigma$ y $\beta$, entonces\\
\begin{align*}
\mu&=E(Y)=\sigma\beta\\
\sigma^2&=V(Y)=\sigma\beta^2
\end{align*}
\begin{tabular}{p{0.900\textwidth} p{0.910\textwidth} }
\includegraphics[width=1\textwidth, center]{captura07.PNG}
\end{tabular}\\
Es una distribución adecuada para modelizar el comportamiento de variables aleatorias continuas con
asimetría positiva. Es decir, variables que presentan una mayor densidad de sucesos a la izquierda de la media que a la derecha.\\ En su expresión se encuentran dos parámetros, siempre positivos, ($\sigma$) y ($\beta$) de los que depende su forma y alcance por la derecha, y también la función Gamma $\Gamma(\sigma)$, responsable de la convergencia de la distribución.






\subsection{La distribución de probabilidad beta}
Se dice que un variable aleatoria \textit{Y} tiene una \textit{distribución de probabilidad beta con parámetros} $\sigma>0$ y $\beta>0$ si y sólo si la funión de densidad de \textit{Y} es
\begin{equation*}
f(y)=\left\lbrace\begin{array}{c} \frac{y^{\alpha-1}(1-y)^{\beta-1}}{B(\sigma,\alpha)},~0\leq~ y ~\leq 1
\\ 0, ~en~ cualquier~ otro ~punto \end{array}\right.
\end{equation*}
donde
\begin{equation*}
B(\alpha,\beta)=\int_{0}^{1}y^{\alpha-1}(1-y)^{\beta-1}dy=\frac{\Gamma(\alpha)\Gamma(\beta)}{\Gamma(\alpha+\beta)}
\end{equation*}
Si $Y$ es una variable aleatoria con distribución beta $\alpha > 0$ y $\beta > 0$, entonces
\begin{align*}
\mu&=E(Y)=\frac{\alpha}{\alpha+\beta}\\
\sigma^2&=V(Y)=\frac{\alpha\beta}{(\alpha+\beta)^2(\alpha+\beta+1)}
\end{align*}
\begin{tabular}{p{0.900\textwidth} p{0.910\textwidth} }
\includegraphics[width=1\textwidth, center]{captura08.PNG}
\end{tabular}\\







\newpage
\subsection{Tabla de resumen}
\begin{center}
%\begin{sideways}


\renewcommand{\arraystretch}{1.8}
\begin{tabular}[t]{|c |c |c |c |}
\hline
Distribucion de Probabilidad & Función de densidad f(y) & Esperanza  & Varianza \\
\hline
Uniforme  & $\frac{1}{\theta_2 - \theta_1}$ & $\mu=E(Y)=\frac{\theta_1 +\theta_2}{2}$&$\sigma^2=V(Y)=\frac{(\theta_2-\theta_1)^2}{12}$\\
\hline

Normal & $f(y)=\frac{1}{\sigma\sqrt{2\pi}}e^{-(y-\mu)^2/(2\sigma^2)}$ & $E(Y)=\mu$ &
$V(Y)=\sigma^2$  \\
\hline

Gamma & $\frac{y^{\sigma-1}e^{-y/\beta}}{\beta^\sigma \Gamma(\sigma)}$ & $\mu=E(Y)=\sigma\beta$ &
$\sigma^2=V(Y)=\sigma\beta^2$\\
\hline


Beta & $ \frac{y^{\alpha-1}(1-y)^{\beta-1}}{B(\sigma,\alpha)}$  & $\mu=E(Y)=\frac{\alpha}{\alpha+\beta}$ & $\sigma^2=V(Y)=\frac{\alpha\beta}{(\alpha+\beta)^2(\alpha+\beta+1)}$\\

\hline
\end{tabular}
%\end{sideways}
\end{center}




\end{document}
